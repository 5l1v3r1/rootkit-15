%%%%%%%%%%%%%%%%%%%%%%%%%%%%%%%%%%%%%%%%%
% University/School Laboratory Report
% LaTeX Template
% Version 3.1 (25/3/14)
%
% This template has been downloaded from:
% http://www.LaTeXTemplates.com
%
% Original author:
% Linux and Unix Users Group at Virginia Tech Wiki
% (https://vtluug.org/wiki/Example_LaTeX_chem_lab_report)
%
% License:
% CC BY-NC-SA 3.0 (http://creativecommons.org/licenses/by-nc-sa/3.0/)
%
%%%%%%%%%%%%%%%%%%%%%%%%%%%%%%%%%%%%%%%%%

%----------------------------------------------------------------------------------------
%   PACKAGES AND DOCUMENT CONFIGURATIONS
%----------------------------------------------------------------------------------------

\documentclass[a4paper,12pt,oneside]{article}
%\usepackage[english]{babel}
\usepackage[utf8]{inputenc}
\usepackage[T1]{fontenc}
\usepackage{lmodern}
%\usepackage{times} % Uncomment to use the Times New Roman font

%----------------------------------------------------------------------------------------
%   DOCUMENT INFORMATION
%----------------------------------------------------------------------------------------

\title{Keylogger/Backdoor Rootkit} % Title

\author{Clemens \textsc{Brunner}\\
        Michael \textsc{Fröwis}} % Author name


\date{\today} % Date for the report

\begin{document}

\maketitle % Insert the title, author and date
\thispagestyle{empty}
\newpage
\tableofcontents
\thispagestyle{empty}

\newpage

\section{Introduction}

In the following we want to explore how to make a linux kernel rootkit. As the definition of a rootkit stats it should run as root and should be hard to detect for users. To give the rootkit real value it has to do something. We decided to go with two very common usecases when it comes to 


\subsection{Kernel Modules}

\subsection{Kernel Rootkits}

\section{Implementation}
\subsection{Keylogging}
\subsection{Backdoor}
\subsection{Hiding}
\subsection{Networking and Activation}

\section{Conclusion}

\end{document}

